\documentclass[a4paper,notitlepage]{article}
\usepackage{pfsstyle}
\title{[DRAFT] Services related on PFS Project (ver 1.pre)}
\author{PFS Project}
\date{2012/10/01}
\begin{document}

\maketitle
\tableofcontents

\section{Abstract}

In this document, internet based services related on PFS Project are summarized 
as of the day of release. 

\section{New account registration}

If you are new to PFS, you will need to request two accounts: 
"SuMIRe/PFS pbworks wiki" (account name is email address) 
and "pfs.ipmu.jp" (you need to select account name). 
Please submit your account request to \tbd 
with email address and preferred account name.

\subsection{SuMIRe/PFS pbworks wiki}

You can request access by yourself from \url{http://sumire.pbworks.com/}, 
with your email address.

Please choose your preferred password on registration. 
After logging in, add your name into member list in the top page. 

\subsection{pfs.ipmu.jp}

You can use every services on this server with only one account
\footnote{Except for some special services, such as git repository (using 
ssh public key authentication).}.
Normally, you can access only via Web (https). 

Account information will be sent via email, please change your password 
from initial one (in email) by following notification email 
\footnote{Also, you can add/change your name (default to account name), 
institution, and photo from web.}. 

\section{SuMIRe/PFS pbworks wiki}

"SuMIRe/PFS PBworks" is the only official wiki system for PFS. 

\subsection{Login and ToC}

After logging in at \url{http://sumire.pbworks.com/}, 
you will get a project index page named 
"Subaru Measurement of Images and Redshifts (SuMIRe)". 
This page contains links to contents in the wiki, such as 

\begin{itemize}
  \item PFS Project Office (General links, including link to telecon indexes)
  \item Documents (Link list to documents)
  \item Meeting, conference, etc.  (Link list to meetings)
  \item Mailing lists (List of avail lists)
  \item PFS working groups (Member list)
\end{itemize}

\subsubsection{Page indexes}

For some continuous meetings, index for each agenda/memo pages are avail.

\begin{itemize}
  \item Technical collaborators' weekly telecon \url{http://sumire.pbworks.com/PFS%20technical%20collaborators%27%20weekly%20telecon}
  \item Systems Engineering group telecon \url{http://sumire.pbworks.com/System-engineer-group-telecon}
  \item Manager group telecon \url{http://sumire.pbworks.com/Manager-group-telecon}
\end{itemize}


\subsubsection{Files}

Also, you can find all files uploaded into this SuMIRe/PFS PBworks from 
\url{http://sumire.pbworks.com/w/browse/#view=ViewAllFiles}.
Some files are categorized into {\bf FOLDERS}, and you can get each list by 
clicking FOLDER name at left side. 

\subsection{Editing manual}

You can find manual for editing PBworks at 
\url{http://usermanual.pbworks.com/}. 


\section{pfs.ipmu.jp - Web access}

\subsection{Public Web Site -- "Fact Sheet"}

This {\bf Fact Sheet} is public open web site (rather for professionals 
not general public) for PFS. 
You can access without any login from \url{http://pfs.ipmu.jp/factsheet/}.

\subsection{Internal Web Site -- https}

Every contents at \url{https://pfs.ipmu.jp/} are project only, and you will 
be required to log in to view pages. 

\subsubsection{Login and user account}

Use your 'account name' (not email address) and 'password'.
For your first time, please follow notification email to change your password 
from an initial one (randomly created). 

You can edir your account information from {\bf LDAP account manipulator} 
service at \url{https://pfs.ipmu.jp/ldap-manip/}, 
such as password, your real name, institution, and photo. 
Also you can view list of all accounts from 

\begin{itemize}
  \item \url{https://pfs.ipmu.jp/ldap-manip/view_all.cgi} (List of existing accounts)
  \item \url{https://pfs.ipmu.jp/ldap-manip/view_allphoto.cgi} (photo directory)
\end{itemize}

\subsubsection{ToC}

When accessing to \url{https://pfs.ipmu.jp/}, you will get newest list of 
contents in this server. 

\paragraph{LDAP account manipulator}

You can view your account setting, list of all avail accounts, and photo 
directory. 
Also, you can edit your account setting (real name, password, institution, 
photo, etc.) from this service.

\paragraph{Document server system}

This is a document strage service.

\paragraph{Issue tracker system (Bugzilla)}

This is the issue tracker for the PFS project. 
Please refer Bugzilla help page at 
\url{http://www.bugzilla.org/docs/tip/en/html/}.

\paragraph{Internal wiki (not official)}

Internal wiki used for pfs.ipmu.jp server administration and scratch. 

\paragraph{pfs.ipmu.jp internal ML}

This internal ML uses mailman, and you will get list of avail lists. 
It depends on settings per each list, you can view registered members, 
view logs of past emails, and also register (or request to register) on 
each list. 

\paragraph{Photo archive}

Photo archive for pfs.ipmu.jp, 
please contact system administrator (\url{pfs@pfs.ipmu.jp}) to put new 
set of photos. 

\paragraph{Etherpad list}

Etherpad is a web-based collaborative real-time editor, 
which you can share (simple) text editor over network without submitting 
nor reloading. 

\paragraph{WebDAV Storage}

This is WebDAV storage, which you can also upload files via clients 
supporting WebDAV protocol (like cadaver). 
You can upload/store/publish any project related files to this space, 
including temporary file exchange. 

\paragraph{Subversion repository and ViewVC}

You can use subversion repositories, please contact organizers of each 
repository for its operational rules. 

Also, by accessing ViewVC, you can check logs of each repository. 

\paragraph{Landfill services}

Some landfill instances would avail. 
(Note: landfill will be used for some testing purpose, but not a real 
operated service.) 

\paragraph{Server status viewer}

For system administration use, you can view system status graph.

\subsection{pfs.ipmu.jp -- git (ssh public key) access}

Git repositories at pfs.ipmu.jp is mostly used for PFS software team.

\subsubsection{Login}

You need to register to gitolite repository manager, even if you have account 
for \url{https://pfs.ipmu.jp/}. Please contact to git administration list 
(email address \tbd) with your ssh public key.

\subsubsection{Access to git repository}

Access to pfs.ipmu.jp port 22 (ssh) with account 'gitolite' 
using your ssh private key. 

Normally, you will be better to configure as following. 
First, add an entry into {\tt .ssh/config} file like 

\begin{verbatim}
host pfsgit
  user gitolite
  hostname pfs.ipmu.jp
  port 22
  identityfile ~/.ssh/id_rsa
\end{verbatim}

and, second access to git repository with 
{\tt git clone ssh://pfsgit/"repository-name"}.

\subsubsection{Repository policy}

\tbd (Will be issued as ICD or SSN)



\end{document}

