\documentclass[a4paper,notitlepage]{article}
\usepackage{pfsstyle}
\title{Services related on PFS Project (ver 3.0)}
\author{PFS Project}
%\date{2012/10/01}
\begin{document}

\maketitle
\tableofcontents

\section{Abstract}

In this document, internet based services related on the PFS Project are summarized 
as of the day of release of this document. 

\section{New account registration}

If you are new to PFS, you will need to request several accounts: 
"SuMIRe/PFS pbworks wiki" (account name is email address), 
"pfs.ipmu.jp" (you need to pick account name), 
"PFS JIRA" (open registration), and 
"GitHub" (organization member for private repositories). 
Submit your account request as following account policy and registration. 

\subsection{Account policy}

Issuing account will depend on your role in the PFS project --- 
technical team member (from each collaborating institute and Subaru) 
or science team member. 

\subsubsection{Technical team member}

You need at least to request an account to log in to \url{https://pfs.ipmu.jp}.
In addition, if you work for software development, you can optionally 
request a member privilege of GitHub: this is needed for you to access 
to private repositories. 
GitHub is also used for an issue tracking system for the development of 
ICS (Instrument Control Software). Software documentation (ics\_doc) 
repository is also on GitHub as a private repository. 

To make these requests, please ask the local site manager at your institute 
to send the following information to \url{pfs@pfs.ipmu.jp}. 

\begin{itemize}
  \item Your contact email address.
  \item Preferred account name for pfs.ipmu.jp.
  \item Photo for photo directory at pfs.ipmu.jp. (JPEG, less than 50kB; width 
    250--300pixel, height 250--350pixel)
  \item (If you are software developers or related workers) Your GitHub account name
\end{itemize}

For the Git repositories operated at GitHub, 
there is one (mostly private) team per institute where the members have 
the admin privilege. 
You can ask a member of the team in your institute to add yourself and 
get the admin privilege to the repositories related to the development 
activities at your institute. Adding you to the team is as 'invitation', 
and you will need to approve an invitation via email. 

Meanwhile, JIRA is used as a management tool for the development of 
the other PFS software packages than ICS, such as DRP (Data Reduction 
Pipeline), ETS (Exposure Targeting Software) and SPT (Survey Planning 
and Tracking software). PFS JIRA (https://pfspipe.ipmu.jp) is open to 
the public, but is operated in an integration of single sign on (SSO) with 
other services (LDAP service of pfs.ipmu.jp), you need to use the same 
account name and password. If you already have an account at pfs.ipmu.jp 
and have not logged in to PFS JIRA before, you need to contact to 
\url{jira@pfs.ipmu.jp} to activate your account in PFS JIRA.


\subsubsection{Science team member}

If you don't have SuMIRe/PFS pbworks wiki account, first request an account 
for SuMIRe/PFS pbworks wiki. 

If you are interested in the PFS software development (DRP, ETS, and SPT), 
access as follows: 
\begin{itemize}
  \item Git repositories are open to the public at \url{https://github.com/Subaru-PFS}
  \item Register yourself to JIRA at \url{https://pfspipe.ipmu.jp} for the development of software packages.
\end{itemize}

\subsection{List of registrations}

\subsubsection{SuMIRe/PFS pbworks wiki}

You can request access by yourself from \url{http://sumire.pbworks.com/} 
("Request access" button at right side), with your email address
\footnote{You also can find "Forgot your passowrd?" link next to the "Log in" 
button to reset your password.}.

Please choose your preferred password on registration. 
After logging in, add your name into member list in the top page. 

\subsubsection{pfs.ipmu.jp (https) --- general}

You can use services on this web server, with only one account for every 
services.

Once your request has granted, 
account information will be sent via email, please change your password 
from initial one (in email) by following notification email 
\footnote{Also, you can add/change your name (default to account name), 
institution, and photo from web.}. 

If you need to reset your password, contact \url{pfs@pfs.ipmu.jp}. 

\subsubsection{GitHub}

PFS has its organization at GitHub as \url{https://github.com/Subaru-PFS}. 
You need to be in one team of this organization for accessing private 
repositories for ICS. 
For full list of repositories, refer GitHub page at 
\url{https://subaru-pfs.github.io/}.

\subsubsection{PFS JIRA}

PFS JIRA (\url{https://pfspipe.ipmu.jp/}) is open to the public, 
so you can register by yourself. 
If you have an account at pfs.ipmu.jp and have not logged in to PFS JIRA 
before, contact \url{jira@pfs.ipmu.jp} to activate your account and use 
the same user name and password as pfs.ipmu.jp. 

\section{Services in detail}

\subsection{SuMIRe/PFS pbworks wiki}

"SuMIRe/PFS PBworks" is the only official wiki system for PFS. 

\subsubsection{Login and ToC}

After logging in at \url{http://sumire.pbworks.com/}, 
you will get a project index page named 
"Subaru Measurement of Images and Redshifts (SuMIRe)". 
This page contains links to contents in the wiki, such as 

\begin{itemize}
  \item PFS Project Office (General links, including link to telecon indexes)
  \item Documents (Link list to documents)
  \item Meeting, conference, etc.  (Link list to meetings)
  \item Mailing lists (List of avail lists)
  \item PFS working groups (Member list)
\end{itemize}

\subsubsection{Page indexes}

For some continuous meetings, index for each agenda/memo pages are avail.

\begin{itemize}
  \item Technical collaborators' weekly telecon \url{http://sumire.pbworks.com/PFS%20technical%20collaborators%27%20weekly%20telecon}
  \item Systems Engineering group telecon \url{http://sumire.pbworks.com/System-engineer-group-telecon}
  \item Manager group telecon \url{http://sumire.pbworks.com/Manager-group-telecon}
\end{itemize}


\subsubsection{Files}

Also, you can find all files uploaded into this SuMIRe/PFS PBworks from 
\url{http://sumire.pbworks.com/w/browse/#view=ViewAllFiles}.
Some files are categorized into {\bf FOLDERS}, and you can get each list by 
clicking FOLDER name at left side. 

\subsubsection{Editing manual}

You can find manual for editing PBworks at 
\url{http://usermanual.pbworks.com/}. 


\subsection{Public Web Site at pfs.ipmu.jp}

Public project information is at \url{http://pfs.ipmu.jp}, 
such as list of meetings, list of publications, and instrument parameters. 

\subsection{Internal Web Site -- pfs.ipmu.jp (https)}

Every contents at \url{https://pfs.ipmu.jp/} are project only, and you will 
be required to log in to view pages. 

If you have any issue on this site, contact administrator 
at \url{pfs@pfs.ipmu.jp}. 

\subsubsection{Login and user account}

Use your 'account name' (not email address) and 'password'.
For your first time, please follow notification email to change your password 
from an initial one (randomly created). 

You can edir your account information from {\bf LDAP account manipulator} 
service at \url{https://pfs.ipmu.jp/ldap-manip/}, 
such as password, your real name, institution, and photo. 
Also you can view list of all accounts from 

\begin{itemize}
  \item \url{https://pfs.ipmu.jp/ldap-manip/view_all.cgi} (List of existing accounts)
  \item \url{https://pfs.ipmu.jp/ldap-manip/view_allphoto.cgi} (photo directory)
\end{itemize}

\subsubsection{ToC}

When accessing to \url{https://pfs.ipmu.jp/}, you will get newest list of 
contents in this server. 

\begin{itemize}
  \item LDAP account manipulator \\
    You can view your account setting, list of all avail accounts, and photo 
    directory. 
    Also, you can edit your account setting (real name, password, institution, 
    photo, etc.) from this service.
  \item List of fuze telecons (also see.~\ref{sec:pfs-fuze})
  \item Content sharing services
    \begin{itemize}
      \item Document server for PFS
      \item Photo archive
      \item WebDAV Storage (see sec.~\ref{sec:pfs-webdav})
    \end{itemize}
  \item Issue tracker and ticketing system
    \begin{itemize}
      \item Issue tracker system -- Bugzilla : 
        for help refer Bugzilla help page at 
        \url{http://www.bugzilla.org/docs/tip/en/html/}.
    \end{itemize}
  \item Temporal sharing services
    \begin{itemize}
      \item Etherpad list : web-based collaborative real-time editor
      \item pastebin
      \item EtherCalc : online spreadsheet
    \end{itemize}
  \item pfs.ipmu.jp internal maillist (ML) :
    web interface of mailman, and you will get list of avail lists. 
    It depends on settings per each list, you can view registered members, 
    view logs of past emails, and also register (or request to register) on 
    each list. 
  \item Internal wiki (not official) : 
    for pfs.ipmu.jp server administration and scratch. 
  \item Server status viewer : 
    for system administration use, you can view system status graph.
\end{itemize}

\paragraph{Photo archive}

Photo archive for pfs.ipmu.jp, 
contact system administrator (\url{pfs@pfs.ipmu.jp}) to put new 
set of photos, after uploading your phots to WebDAV. 


\paragraph{WebDAV Storage}
\label{sec:pfs-webdav}

You can upload files via clients supporting WebDAV protocol, like cadaver on 
Linux and MacOS. 
You can upload/store/publish any project related files to this space, 
including temporal file exchange. 

Please refer a page in the internal wiki (\url{https://pfs.ipmu.jp/wiki/System/webdav}) 
for how to connect to WebDAV storage.

\subsubsection{Landfill services}

Some landfill instances would avail. 
(Note: landfill will be used for some testing purpose, but not a real 
operated service.) 


\subsection{Fuze conference system}
\label{sec:pfs-fuze}

PFS project uses Fuze service for teleconference. 
You will get invitation for meetings by email, which has on-line (VoIP) 
access URL and phone numbers with introductions to join. 
For accessing on-line, you will need to install clients avail from 
\url{https://www.fuze.com/download}. 

You can check list of planned teleconferences from 
\url{https://pfs.ipmu.jp/fuzelist/}. 


\end{document}

