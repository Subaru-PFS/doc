\documentclass[a4paper,notitlepage]{article}
\usepackage{pfsstyle}
\title{Services related on PFS Project (ver 3.0)}
\author{PFS Project}
%\date{2012/10/01}
\begin{document}

\maketitle
\tableofcontents

\section{Abstract}

In this document, internet based services related on the PFS Project
are summarized as of the day of release of this document.

\section{Overview}

Below are a few tools and resources that are used in the PFS
project. You may need account(s) and access to some or all of them
depending on your roles and contributions to the project.

\begin{itemize}
\item SuMIRe/PFS PBworks wiki
\item pfs.ipmu.jp
\item PFS JIRA
\item GitHub
\end{itemize}

A project management tool ``Atlassian JIRA'' is used in the PFS
project for the issue tracking in the development of the software
packages: ICS (Instrument Control Software), such as DRP (Data
Reduction Pipeline), ETS (Exposure Targeting Software) and SPT (Survey
Planning and Tracking software), and database. The contents on PFS
JIRA \url{https://pfspipe.ipmu.jp/jira/secure/Dashboard.jspa} are open
to the public but through Single Sign On (SSO) integrated with other
services at pfs.ipmu.jp (e.g. LDAP service).

The PFS project also has an organization at GitHub titled as
``Subaru-PFS'' \url{https://github.com/Subaru-PFS} for the software
coding works and documentation. The contents therein are open to the
public except for those in the private repositories.

\section{Account policy and registration procedures}

First of all, you need an account for access to SuMIRe/PFS PBworks
wiki page. You can go to \url{http://sumire.pbworks.com} and requet an
account.

After you have access to this SuMIRe/PFS PBworks wiki, your needs of
other accounts and services depend on your role in the PFS project as
explained in \S~\ref{subsec:techmem} and  \S~\ref{subsec:scimem}.

\subsection{Technical team member}
\label{subsec:techmem}

If you are committing to works for the instrument development as a
member in the technical team of the PFS project, you need to get an
account to use services of pfs.ipmu.jp (https) at least. Optionally
you need access to PFS JIRA and a member privilege of PFS GitHub if
you work on software development. Ask the local site manager at your
institution to send the following information to
\url{pfsprojectoffice@ipmu.jp}:

\begin{itemize}
\item Your contact email address
\item Preferred account name
\item Photo for photo directory at pfs.ipmu.jp. (JPEG, less than 50kB;
  width 250--300pixel, height 250--350pixel)
\item (If you will work on software development)
\item Request of access to PFS JIRA (If you need to push/pull Git
  repositories) Your GitHub account
\end{itemize}

Once your request is granted, the account information for pfs.ipmu.jp
is emailed, so you must change your password from the initial one
delivered by the email\footnote{Also, you can add/edit your personal
  information such as your affiliation.}.  If you need to reset your
password, contact \url{pfs@pfs.ipmu.jp}.

\subsection{Science team member}
\label{subsec:scimem}

If you are not working for the instrument development, you are
categorized as a science team member here, and usually you do not need
any other accounts. But depending on your roles and contributions to
the studies of PFS science cases and survey planning, you may need an
account to use services at pfs.ipmu.jp (https). In this case, you
should send the following information to
\url{pfsprojectoffice@ipmu.jp}:

\begin{itemize}
\item Your contact email address
\item Preferred account name
\item Photo for photo directory at pfs.ipmu.jp (JPEG, less than 50kB;
  width 250--300pixel, height 250--350pixel)
\end{itemize}
  
In addition, if you work on PFS software development, you should
follow the procedure as below:

\begin{itemize}
\item Contact a local site manager at your institute in the PFS
  technical team and formalize your commitment(s) to the software
  development.
\item Send an e-mail to jira@pfs.ipmu.jp and request access to PFS
  JIRA, if you need to actively participate in discussions by file
  tickets, sending comments, and so on.
\item Send your GitHub account by e-mail to
  \url{pfsprojectoffice@ipmu.jp}, if you need to push/pull git
  repositories for software coding works and documentations.
\end{itemize}

\subsection{Notes and tips for the registration process}

\begin{itemize}
\item PFS JIRA and PFS GitHub are open to the public, so you can read
  the contents except for those in the private repositories, but you
  cannot make any commitments therein unless you follow the above
  procedure.
\item If you already have a pfs.ipmu.jp account but cannot access to
  PFS JIRA, you should send an e-mail to \url{jira@pfs.ipmu.jp}.
\item In the PFS GitHub organization, there is one (mostly-private)
  team per institute where the members are given admin privilege. So
  instead of sending an e-mail to \url{pfsprojectoffice@ipmu.jp} (or
  \url{github@pfs.ipmu.jp}), you can also ask the members in your
  institute to add yourself with admin privilege given to repositories
  related to your institute. You then receive an ``invitation'' by
  e-mail, so you need to approve it following the instruction therein.
\end{itemize}

\subsubsection{PFS JIRA}

PFS JIRA (\url{https://pfspipe.ipmu.jp/}) is open to the public, 
so you can register by yourself. 
If you have an account at pfs.ipmu.jp and have not logged in to PFS JIRA 
before, contact \url{jira@pfs.ipmu.jp} to activate your account and use 
the same user name and password as pfs.ipmu.jp. 

\section{Details of the services}

\subsection{SuMIRe/PFS pbworks wiki}

"SuMIRe/PFS PBworks" is the only official wiki system for PFS. 

\subsubsection{Login and ToC}

After logging in at \url{http://sumire.pbworks.com/}, 
you will get a project index page named 
"Subaru Measurement of Images and Redshifts (SuMIRe)". 
This page contains links to contents in the wiki, such as 

\begin{itemize}
  \item PFS Project Office (General links, including link to telecon indexes)
  \item Documents (Link list to documents)
  \item Meeting, conference, etc.  (Link list to meetings)
  \item Mailing lists (List of avail lists)
  \item PFS working groups (Member list)
\end{itemize}

\subsubsection{Page indexes}

For some continuous meetings, index for each agenda/memo pages are avail.

\begin{itemize}
  \item Technical collaborators' weekly telecon \url{http://sumire.pbworks.com/PFS%20technical%20collaborators%27%20weekly%20telecon}
  \item Systems Engineering group telecon \url{http://sumire.pbworks.com/System-engineer-group-telecon}
  \item Manager group telecon \url{http://sumire.pbworks.com/Manager-group-telecon}
\end{itemize}


\subsubsection{Files}

Also, you can find all files uploaded into this SuMIRe/PFS PBworks from 
\url{http://sumire.pbworks.com/w/browse/#view=ViewAllFiles}.
Some files are categorized into {\bf FOLDERS}, and you can get each list by 
clicking FOLDER name at left side. 

\subsubsection{Editing manual}

You can find manual for editing PBworks at 
\url{http://usermanual.pbworks.com/}. 


\subsection{Public Web Site at pfs.ipmu.jp}

Public project information is at \url{http://pfs.ipmu.jp}, 
such as list of meetings, list of publications, and instrument parameters. 

\subsection{Internal Web Site -- pfs.ipmu.jp (https)}

Every contents at \url{https://pfs.ipmu.jp/} are project only, and you will 
be required to log in to view pages. 

If you have any issue on this site, contact administrator 
at \url{pfs@pfs.ipmu.jp}. 

\subsubsection{Login and user account}

Use your 'account name' (not email address) and 'password'.
For your first time, please follow notification email to change your password 
from an initial one (randomly created). 

You can edir your account information from {\bf LDAP account manipulator} 
service at \url{https://pfs.ipmu.jp/ldap-manip/}, 
such as password, your real name, institution, and photo. 
Also you can view list of all accounts from 

\begin{itemize}
  \item \url{https://pfs.ipmu.jp/ldap-manip/view_all.cgi} (List of existing accounts)
  \item \url{https://pfs.ipmu.jp/ldap-manip/view_allphoto.cgi} (photo directory)
\end{itemize}

\subsubsection{ToC}

When accessing to \url{https://pfs.ipmu.jp/}, you will get newest list of 
contents in this server. 

\begin{itemize}
  \item LDAP account manipulator \\
    You can view your account setting, list of all avail accounts, and photo 
    directory. 
    Also, you can edit your account setting (real name, password, institution, 
    photo, etc.) from this service.
  \item List of fuze telecons (also see.~\ref{sec:pfs-fuze})
  \item Content sharing services
    \begin{itemize}
      \item Document server for PFS
      \item Photo archive
      \item WebDAV Storage (see sec.~\ref{sec:pfs-webdav})
    \end{itemize}
  \item Issue tracker and ticketing system
    \begin{itemize}
      \item Issue tracker system -- Bugzilla : 
        for help refer Bugzilla help page at 
        \url{http://www.bugzilla.org/docs/tip/en/html/}.
    \end{itemize}
  \item Temporal sharing services
    \begin{itemize}
      \item Etherpad list : web-based collaborative real-time editor
      \item pastebin
      \item EtherCalc : online spreadsheet
    \end{itemize}
  \item pfs.ipmu.jp internal maillist (ML) :
    web interface of mailman, and you will get list of avail lists. 
    It depends on settings per each list, you can view registered members, 
    view logs of past emails, and also register (or request to register) on 
    each list. 
  \item Internal wiki (not official) : 
    for pfs.ipmu.jp server administration and scratch. 
  \item Server status viewer : 
    for system administration use, you can view system status graph.
\end{itemize}

\paragraph{Photo archive}

Photo archive for pfs.ipmu.jp, 
contact system administrator (\url{pfs@pfs.ipmu.jp}) to put new 
set of photos, after uploading your phots to WebDAV. 


\paragraph{WebDAV Storage}
\label{sec:pfs-webdav}

You can upload files via clients supporting WebDAV protocol, like cadaver on 
Linux and MacOS. 
You can upload/store/publish any project related files to this space, 
including temporal file exchange. 

Please refer a page in the internal wiki (\url{https://pfs.ipmu.jp/wiki/System/webdav}) 
for how to connect to WebDAV storage.

\subsubsection{Landfill services}

Some landfill instances would avail. 
(Note: landfill will be used for some testing purpose, but not a real 
operated service.) 


\subsection{Fuze conference system}
\label{sec:pfs-fuze}

PFS project uses Fuze service for teleconference. 
You will get invitation for meetings by email, which has on-line (VoIP) 
access URL and phone numbers with introductions to join. 
For accessing on-line, you will need to install clients avail from 
\url{https://www.fuze.com/download}. 

You can check list of planned teleconferences from 
\url{https://pfs.ipmu.jp/fuzelist/}. 


\end{document}

