\documentclass[12pt]{article}
\usepackage{graphicx}
%
\usepackage{color}
\usepackage[colorlinks=true, citecolor=blue, linktocpage=true]{hyperref} % glossary colour is set in style
\usepackage{xspace}
\usepackage{import}
\usepackage[titletoc,title]{appendix}
\usepackage[margin=1.0in,nohead,nofoot]{geometry}
%\usepackage[titletoc,title]{appendix}
%\usepackage[style=authoryear, uniquename=false, backend=bibtex]{biblatex}
%\usepackage{journals}
\usepackage[acronym,nonumberlist,toc,xindy]{glossaries}
%\bibliography{XXX}
%------------------------------------------------------------------------------

\renewcommand{\c}{\textit{c.}\xspace}
\newcommand{\cf}{\textit{cf.}\xspace}
\newcommand{\eg}{\textit{e.g.}\xspace}
\newcommand{\etc}{\textit{etc.}\xspace}
\newcommand{\ie}{\textit{i.e.}\xspace}
\newcommand{\nb}{\textit{n.b.}\xspace}
\newcommand{\Nb}{\textit{N.b.}\xspace}

%
% Convenient (but unnecessary) abbreviations
%
\newcommand{\AGC}{\gls{AGC}\xspace}
\newcommand{\ICS}{\gls{ICS}\xspace}
\newcommand{\IIC}{\gls{IIC}\xspace}
\newcommand{\MCS}{\gls{MCS}\xspace}
\newcommand{\MHS}{\gls{MHS}\xspace}
\newcommand{\FPS}{\gls{FPS}\xspace}
\newcommand{\opDB}{\gls{opDB}\xspace}
\newcommand{\PFI}{\gls{PFI}\xspace}
\newcommand{\PFS}{\gls{PFS}\xspace}
\newcommand{\SPS}{\gls{SPS}\xspace}
\newcommand{\TBD}{\gls{TBD}\xspace}

\newcommand{\jupyter}{\gls{jupyter}\xspace}
\newcommand{\ssh}{\gls{ssh}\xspace}

%------------------------------------------------------------------------------

\newcommand{\XXX}[1]{\textbf{XXX: #1}}
\newcommand{\appRef}[1]{Appendix \ref{sec:#1}} % n.b. same prefix (sec:) as \secRef
\newcommand{\secRef}[1]{Section \ref{sec:#1}}

%------------------------------------------------------------------------------

\import{.}{initializeGlossary}
\import{../misc/}{PFS_Abbreviations}
\import{.}{glossaryEntries}

%------------------------------------------------------------------------------

\begin{document}
\title{\acrshort{PFI} Operations at Subaru}
\author{Robert Lupton}
%\date{1996-03-06}
\maketitle

%------------------------------------------------------------------------------

I have been asked to think about how the complete \PFS instrument behaves at the summit, with particular
reference to the hardware and software components associated with the cobra system:
\begin{itemize}
\item\AGC
\item\MCS
\item\FPS
  Uses \gls{cobraCharmer} to control the \gls{FPGA} and \gls{cobraOps} to plan moves. 
\item\PFI
\item\IIC
  The high-level command and control actor
\end{itemize}

These systems are all commanded by the \ICS, and commands are passed between subsystems
using the \MHS. The interface to the telescope is through \gls{Gen2}, specifically via an \MHS
actor called \gls{gen2actor}.

Some of the discussion is based on the
documents by Yabe-san on \href{https://github.com/Subaru-PFS/doc/tree/master/operation}{operations} and
Moritani-san on 
\href{https://github.com/Subaru-PFS/ics\_doc/blob/master/commissioning/commissioning\_details.pdf}{commissioning} and this document should not be taken to supersede either of these;  rather it is intended to clarify the
scope of the software that must be provided as part of \PFS's successful integration and test.  Wherever
possible this software should be written, tested, and \textit{used} before the hardware is shipped to Hawai`i.
This has not always been the case for previous instruments (\eg CHARIS).

This document does not discuss how measurements (\eg bias exposures, the seeing and transparency) are logged
to the \opDB, although this is of critical importance.

Some of the analysis code discussed here may belong in the \MHS actors, but some of it will probably
live outside the core operation software (an example is the rastering code described in \secRef{IIC}).

\section{Relationship to Subaru}

In routine operations we expect that \PFS will be run from \gls{Gen2} like any other Subaru instrument,
but while we are commissioning the system we will need to troubleshoot hardware problems and debug
the software.  We do not expect that this will be needed in routine operations, and we therefore do
not plan to make the debugging tools available \textit{as part of the routine mountain top environment};
however we will provide documentation and support, and where appropriate training, to make these tools
usable by Subaru staff.

In particular, we believe that we need to be able to commission \PFS without using any \gls{Gen2} visualisation
tools as they are designed to support routine operations rather than the particular needs of the \PFS project.
As we identify diagnostics, whether numerical or visual, that Subaru would like to adopt for long-term
\PFS operations we will help make them available via \gls{Gen2}.  As emphasized in \secRef{availability}
the \PFS project expects to have developed the basic tools before shipping, which means that this
collaboration with Subaru should be able to start before the hardware arrives on the Mountain.

An exception is the \gls{STS}; all error conditions detected by the \PFS instrument and deemed worthy
of attention will be made available to Subaru using the \gls{STS} system.  The analysis of
internal \PFS state and decisions about what is indeed an error will be made by the \MHS \texttt{Alerts Actor}.

\section{PFS's Software Environment at Subaru}

Because this document's scope covers troubleshooting at the summit we expect a significant amount of ad-hoc
analysis, which means that we need a flexible and powerful way of interacting with the data (engineering and
scientific) which we are taking.  The \PFS software is built around python, and all tools needed on the summit
must therefore expose a python interface unless granted an explicit exemption by the Project Manager or
his nominated designate.

For work in Hilo and on the summit there are no restrictions on access to most \PFS computers and in
particular all the disks will be visible to the \PFS network.  We will need the same level of access from \eg
Princeton, IPMU, and from other arbitrary locations (\eg hotels), and exactly how this will be achieved needs
to be negotiated.  We expect that the Subaru \gls{VPN} will be sufficient, as the \PFS
configuration options no longer route all traffic through Japan.

I propose that we standardize on using \gls{pyQt5} for status displays (as already adopted in the \SPS
work at \gls{LAM}) and
\jupyter as the interface to the data (see \appRef{jupyter}).

There are two ways to work remotely with the \PFS data, and both should be supported (both will
make use of the \gls{VPN} to connect, possibly via \ssh tunnels):
\begin{itemize}
\item Using a local browser to connect to a \jupyter server running
  on \PFS machines in Hawai`i
\item Connecting from a local machine to a \gls{VNC} instance running in Hawai`i
\end{itemize}
By a \textit{local} program I mean one running on \eg on your laptop.

We anticipate that largely pre-defined displays (such as the guide signals from the \AGC) will make
use of \glspl{VNC}, while ad-hoc analysis using \jupyter is more likely to connect to a server
using \ssh tunnels.  Image and graphical displays are more problematic; it is possible to display
these using \jupyter inside a browser, or some people may choose to run the displays within
a \gls{VNC} session and command them via a local \jupyter session.  In order to support both of
these modes of image display we propose using the device-agnostic image display protocol ``afwDisplay''
provided by the LSST stack;\footnote{But refactored in such a way that it does not depend on other LSST code.}
backends include \gls{ginga}, \gls{ds9}, and \gls{matplotlib}.

We also need to do some refactoring of our current \PFS instrument software to create libraries
which can be called from other places, \eg analysis code.

\subsection{Availability of PFS Software}
\label{sec:availability}

Telescope time is extremely valuable, and \PFS has a duty to ensure that the hardware and software
is as ready as it can be before we ship to Hawai`i, and in particular before any daytime or nighttime
operations on the telescope.

Some of the functionality described in the document cannot be validated before on sky operations,
but almost all of the tooling and plumbing can be written and tested as part of the instrument
development, and the \PFS project is committed to so doing.

\section{AGC}
\label{sec:AGC}

In many ways the \AGC is the simplest of the \PFS systems, although it has one of the largest interfaces to
the telescope, as while the logic of acquiring fields and focussing can be carried out entirely by \PFS,
the actual
commands to the telescope must pass through \gls{Gen2}\footnote{With the exception of the offset signal
  which is passed by \gls{MAC} directly to \gls{MLP1}.}.  It is essential that \PFS have the ability to do its
own field acquisition as its capabilities are rather different from previous Subaru instruments:
\begin{itemize}
\item \PFS has much more stringent requirements on guiding than imagers
\item \PFS has 6 guide cameras, and will guide on more than one star
\item \PFS requires the ability to update the rotator angle based on guide stars
\item \PFS will generate lists of guide stars while choosing objects to be observed;  the final
  list will be determined only when the mapping of objects to cobras has been made.
\item \PFS will generate a focus signal, rather than expecting the observer to take
  manual focus sweeps
\end{itemize}
and we will need to be able to debug and polish this functionality on the sky.

We do not yet know how much of field acquisition will be done by the \PFS instrument during
science operations, especially in open-use time, and as appropriate some of these functionalities may be
transferred to, or duplicated by \gls{Gen2}, presumably learning from \PFS's experience.

\subsection{ISR and Image Processing}

We expect to need some sort of simple \gls{ISR} for the \AGC, probably just bias and flat fielding using
flats derived from sky frames.  Both the generation of calibration products and their application could
be easily done using the \gls{HSC}/\gls{LSST} code that we already use at Subaru to process \gls{HSC} data.

Tasks include
\begin{itemize}
  \item Perform simple \gls{ISR} on the frames.
    The \gls{ISR} is also responsible for estimating the image variance, either per-pixel or per camera.
  \item Estimate and subtract the sky
  \item Carry out object detection and centroiding.  The defocus
    is probably not an issue for any of these operations
  \item Estimate the fluxes of objects in the field
  \item Measure suitably-weighted second moments for detected objects. The weighting function
    will probably be a suitably chosen Gaussian (but not chosen per-object as we need to
    calibrate how we estimate the defocus given these moments). It is possible that we will
    need to crudely estimate the \gls{PSF} in order to set the weighting function.
\end{itemize}

\subsection{Field Acquisition}
\label{sec:FieldAcquisition}

After the telescope is at the nominal field position the \AGC will take an exposure.  It must
be able to:
\begin{itemize}
  \item Display the post-sky subtracted images as a mosaic of all six chips arranged
    with the correct relative orientation and rotation.
  \item Overlay the positions of detected objects
  \item Overlay the expected positions of guide stars at the epoch of the exposure
\end{itemize}

Note that the \AGC needs access to the guide star catalogue being used.  It seems likely that we will be able
to use \gls{Gaia}, but it is the job of the software that determines the scientific targets to also provide a
list of possible guide stars, probably as
part of the \gls{pfsDesign} file.  This catalogue must also provide estimates of the flux of the stars in the
\AGC filter passband, and allowing for the devices' sensitivity curves.  It is likely that these colour
terms will be determined on the sky; high accuracy is not required, so \eg \gls{GaiaG} and \gls{GaiaBPRP}
would probably suffice.

\subsubsection{Transparency/Seeing}

Under poor conditions we will need to use the results of this analysis to adjust the \AGC exposure
times.

We expect that the second moments will be sufficient to characterise the seeing, and the point-source
fluxes will be sufficient to measure the transparency.

\subsubsection{Offset/Rotation}

We need to solve for the offset between the measured and expected positions.  We expect the offset
to be small, but it seems unwise to assume that this is always the case --- however, we should expect
that it \textit{will} be small enough that we do not need a blind solver.  Once again,
the \gls{HSC}/\gls{LSST} code provides this functionality.

We need to be able to display the measured offset vectors as a function of the position of the stars
in the focal plane.

Once we know the offsets in each of the guide cameras in which we have detected stars we can
calculate the scale/rotation change that must be applied via \gls{Gen2}.  Discussing how
this information transfer is achieved is outside the scope of this document.

\subsubsection{Focus and Tilt}

Subaru expects to use code developed by Yoko Tanaka \XXX{reference} to optimize the
alignment of the PFI+WFC assembly against the telescope primary mirror in translation and tilt, based on the
shapes of defocused stellar images on the AG cameras.  Depending on how well the resulting lookup
tables work we may not need to actively control these degrees of freedom.

Every exposure, providing there are enough stars detected, can be used to estimate the focus and tilt
of the \PFI focal plane.  The stars used to make this measurement need not be in the guide catalogue.

We need to be able to overlay the measured second-moments (probably as ellipses) on the guide images,
and plot the image sizes as a function of position on the \PFI focal plane.  Because the distance
from optimal focus
is not measured per-object the visualisation of the piston/tilt is \TBD.

\Nb if the focus shift is due to changes in the distance from the primary mirror to the \PFI then
then focussing will also correct scale errors.  If it is due to changes in the radius of curvature
of the primary then focussing will change the scale.  The \AGC analysis code needs to monitor this
and warn the operator if the scale has changed.

With the second moments in hand we need to calculate the defocus and tilt measurements that must be applied
via \gls{Gen2}.  Discussing how this information transfer is achieved is outside the scope of this document.

%------------------------------------------------------------------------------
%------------------------------------------------------------------------------

\subsection{Guiding}

Once the telescope is locked onto the guide stars the \AGC still needs to provide status displays
as well as sending offsets to \gls{MLP1}.

\subsubsection{Per Guide-Image Displays}

After every \AGC readout all the processing and displays described in \secRef{FieldAcquisition} need
to be available taking into account \gls{ROI} readout.  For example, rather than displaying
all six cameras in the proper orientation we need to display all the chosen guide stars (suitably enlarged)
in their proper orientation and relative position.  The details are \TBD; for example the stars
from each chip should probably be grouped together with the correct relative positions, but the
per-chip groups should be separated using a different scaling.

\subsubsection{Tracking Performance through a PFS exposure}

We also need plots to track the performance of the system.  For example, we will want to see the variation
of the focus with time, broken down suitably by azimuthal angle to show tilts.  Because we will be running
a \gls{Kalman filter} to track the focus, the display will need to show the filtered values too.  Other
examples are the seeing, background, and transparency as a function of time, updated after every \AGC read.

\subsection{Debugging}

All the data described above needs to be available in convenient ways within the \jupyter environment.  In
particular, this means that images need to be available via the \gls{visit} number and an \AGC sequence number
within the visit rather than via a path into the file system.

The catalogues and derived quantities (seeing, fluxes) will also need to be available within \jupyter, and
in standard formats (\eg \gls{pandas} dataframes) and made available to \gls{Gen2} as desired, using
formats \TBD.

\section{MCS}
\label{sec:MCS}

The \MCS is responsible for taking and analysing images of the back-illuminated PFI science and fiducial
fibres.  It is not responsible for understanding or even knowing of the existence of cobras; that is left to the \FPS.

\Nb the current \MCS code does, in fact, assign cobraIds to the spots that it detects, but this
is problematic because:
\begin{itemize}
\item it needs to know the geometry of the cobras, both within the \PFI and arm-lengths, stop positions, \etc.
\item it needs access to the \gls{pfsDesign} data to know which cobras are illuminated
  (\ie not behind spots or broken)
\item it needs to know the distortion maps from sky to \MCS coordinates
\end{itemize}

\subsection{ISR and Image Processing}

As every image is taken we need to:
\begin{itemize}
  \item Perform simple \gls{ISR} on the frame if necessary
  \item Estimate and subtract the background, if necessary
  \item Carry out object detection and centroiding, including error analysis
\end{itemize}

If the image quality is poor, as it was during the initial commissioning run, the centroiding will
need to be more sophisticated, \eg estimating the \gls{PSF} as a function of position, correlating
the image with the \gls{PSF}, and then estimating centroids on the resulting image.  It is
acceptable to only support this sort of analysis in the \jupyter notebooks.

\subsection{Visualisation}

We need to be able to:
\begin{itemize}
\item Display the entire image
\item Overlay the measured centroids of all detected objects
\end{itemize}

\section{FPS}

Although this document is not normative for the \FPS's required functionality, as a reminder
the \FPS must:
\begin{itemize}
\item
  identify cobras from the centroids provided by the \MCS.  This may
  require information about the last known position of the fibres
\item
  link together detections of the same fibre, as there can be more than
  one when we are strobing the illuminator
\item
  Fit arcs of circles to sets of centroids (either from one strobed image or from
  a set of \MCS images) in the case that only one motor is being run.  Note that
  this information must come from the \opDB.
\item
  Update the \glspl{motor map}
\end{itemize}

\subsection{Visualisation}

The \FPS need not provide visualisation of the pixel-level \MCS data.
We need to be able to:
\begin{itemize}
\item Display the measured centroids of all detected objects linked together by cobraId (\eg colour)  
\item Overlay the expected centroids, transformed using the nominal distortion maps
  into the \MCS coordinate system.
  This information comes from the \gls{pfsDesign} file, and the distortion library. The
  source of truth for the fiducial fibres is \TBD
\item Plot the vector field of derived offsets of the fibres from their expected positions,
  including information about which fibres are supposed to be present but aren't detected,
  which are behind spots, and which are broken and should not be detectable.
  We probably need the ability to only plot the fiducial fibres.
\item
  Overlay the best-fit circles when appropriate
\end{itemize}

\begin{itemize}
\item Display the data being used to update the \gls{motor map}
\item Overlay the current (old) \gls{motor map} and proposed new one
\item Show the convergence of each fibre, either individually or as a full-focal-plane graphic
\end{itemize}

\section{IIC}
\label{sec:IIC}

While the \IIC is not directly responsible for any part of the fibre system it is a natural home
for some high-level analysis.  In some cases this code may ultimately live elsewhere.

\subsection{Analyzing Fibre Flux}

As soon as we start an \SPS integration we can check how consistent the data is with the
results from \FPS.  For all objects with a significant and known flux (\eg telluric standards) we
can extract the flux from the \gls{NIR} arm and compare it to the expected value based on the
seeing/transparency values coming from the \AGC and the known cobra pointing errors from the \FPS.

\subsection{Analysing Full-PFI Raster Sequences}

During commissioning we will need to tie the \PFI and \AGC coordinate systems together.   Depending
on how stable the system proves to be we may need to repeat this analysis from time to time.

We will target a dense star field, and take a set of short \SPS exposures with small offsets
between each (\eg a $n\times n$ grid of 1arcsec offsets centred at the nominal position).  This
will probably be driven from a \gls{Gen2} script; $n$ should be odd.

For each pointing we need to:
\begin{itemize}
\item Read out the spectrographs, possibly only the \gls{NIR} arms
\item Extract the flux from each object
\end{itemize}

When all the data has been taken:
\begin{itemize}
\item Retrieve the guide star positions from the \AGC images, and measure the raster offsets
\item Analyse the fluxes to find the centres of the objects
\item Generate an image of all the objects, created from the $n\times n$ rasters, with each object
  at the correct relative position based on the \gls{pfsConfig} file, in the coordinate systems
  of the centre of the dither.
\item Overlay the measured and expected positions
\item Generate a vector image of the offsets between measured and expected positions
\end{itemize}

Update the offsets between the \AGC and \MCS coordinate systems (scale, offset, rotation; or higher
order terms if necessary which seems unlikely). Update the \MCS's mapping from sky to MCS pixel
coordinates if needs be.

If this analysis is carried out on a science field, calculate the parameters needed to reconfigure
the \PFI and/or adjust the telescope's offset/rotation to maximise the \gls{SNR}, and pass these
parameters to \gls{Gen2}.

%------------------------------------------------------------------------------

\begin{appendices}

\section{Jupyter}
\label{sec:jupyter}

\XXX{Write me.  Describe image display and plotting options}

\end{appendices}

%------------------------------------------------------------------------------
%
% Print acronyms/glossaries
%
\printglossary[type=\acronymtype]
\printglossary

\end{document}
